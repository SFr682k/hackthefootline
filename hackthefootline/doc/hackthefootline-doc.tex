%%	This is file 'hackthefootline-doc.tex', Version 2017-08-04
%%	Copyright 2017 Sebastian Friedl <sfr682k@t-online.de>
%% 
%%	This work may be distributed and/or modified under the conditions of the LaTeX Project
%%	Public License, either version 1.3c of this license or (at your option) any later version.
%%	The latest version of this license is available at
%%		http://www.latex-project.org/lppl.txt
%%	and version 1.3c or later is part of all distributions of LaTeX version 2008-05-04 or later
%%
%%	This work has the LPPL maintenace status 'maintained'.
%%	The current maintainer of this work is Sebastian Friedl.
%%
%%	This work consists of the files hackthefootline.sty and hackthefootline-doc.tex
%%
%%	-------------------------------------------------------------------------------------------
%%
%%	The hackthefootline package is a tool providing arbitrary footline selection for
%%	LaTeX beamer's standard themes (other themes may also work, of course)
%%
%%	-------------------------------------------------------------------------------------------
%%
%%	Please report bugs and other problems as well as suggestions for improvements
%%	to my email address (sfr682k@t-online.de).
%%
%%	-------------------------------------------------------------------------------------------
%%
%%	Acknowledgement:
%%	The basic layout of the two- and three-column footline is based on the original LaTeX
%%	beamer split and infolines outer themes written by Till Tantau.
%%
%%	-------------------------------------------------------------------------------------------

% !TeX spellcheck = en_GB

% !TeX document-id = {681db40e-7a84-4428-b4f4-84e230e3ba79}
% !TeX program=lualatex
% !TeX TXS-program:compile=txs:///lualatex/[--shell-escape]


\documentclass[11pt]{ltxdoc}

\usepackage[charter]{mathdesign}
\usepackage[no-math]{fontspec}
\usepackage{polyglossia}
\setdefaultlanguage{english}

\usepackage{csquotes}
\usepackage{hologo}
\usepackage{hyperref}
\usepackage{minted}
	\newcommand{\ltxcmd}[1]{\mintinline{LaTeX}{#1}}
\usepackage[english]{selnolig}
\usepackage{tikz}

\parindent 0pt

\setmainfont{Noto Serif}
\setsansfont[Scale=MatchLowercase]{Source Sans Pro}
\setmonofont[Scale=MatchLowercase]{OCR A Extended}[AutoFakeBold=1.75,AutoFakeSlant=0.225]

\usepackage[left=4.50cm,right=2.75cm,top=3.25cm,bottom=2.75cm,nohead]{geometry}

\hyphenation{}


\title{The \texttt{\bfseries hackthefootline} package \\ {\large\url{https://github.com/SFr682k/hackthefootline}}}
\author{Sebastian Friedl \\ \href{mailto:sfr682k@t-online.de}{\ttfamily sfr682k@t-online.de}}
\date{2017/08/04}

\hypersetup{pdftitle={The hackthefootline package},pdfauthor={Sebastian Friedl}}

\begin{document}
	\maketitle
	\thispagestyle{empty}
	
	\begin{center} \itshape
		Dedicated to all the \LaTeX\ beamer users \dots
		
		\medskip
		\dots who load two different themes in the same presentation \\ or redefine commands to obtain the desired footline \dots
		
		\medskip
		\dots and the other guys just using \mintinline{LaTeX}{\setbeamertemplate{footline}}
	\end{center}
	
	\medskip
	\begin{abstract}
		\hspace{-1.5em}%
		Provides arbitrary footline selection for \LaTeX\ beamer's standard themes (and maybe other themes) by \enquote{hacking} it. \\
		In this content, \enquote{hacking} means to redefine the footline provided by theme with one of the footline templates provided by this package.
	\end{abstract}
	
	
	\tableofcontents
	
	\clearpage
	


	\subsection*{Dependencies and other requirements}
	\addcontentsline{toc}{subsection}{Dependencies and other requirements}
	\emph{Load the \texttt{hackthefootline} package only in \LaTeX\ beamer documents!} \\
	The package itself depends on the following other packages:
	
	\medskip
	\DescribeMacro{appendixnumber-}\DescribeMacro{beamer}
	The \texttt{appendixnumberbeamer} package implements a simple solution for appendix frames not being calculated into the total number of frames
	
	\medskip
	\DescribeMacro{etoolbox}
	Provides access on \hologo{eTeX} primitives
	
	\medskip
	\DescribeMacro{calc}
	Required for basic calculations

	\medskip
	\DescribeMacro{numprint}
	Prints out numbers, counters and lengths
	
	\bigskip
	The dependencies can be removed completely by passing some options to the package. See section \ref{packageoptions} for further details.
	
	
	\subsection*{License}
	\begin{small}
		\addcontentsline{toc}{subsection}{License}
		\textcopyright\ 2017 Sebastian Friedl
		
		\smallskip
		This work may be distributed and/or modified under the conditions of the \LaTeX\ Project Public License, either version 1.3c of this license or (at your option) any later version.
		
		\smallskip
		The latest version of this license is available at \url{http://www.latex-project.org/lppl.txt} and version 1.3c or later is part of all distributions of \LaTeX\ version 2008-05-04 or later.
		
		\smallskip
		This work has the LPPL maintenace status \enquote*{maintained}. The current maintainer of this work is Sebastian Friedl. \\
		This work consists of the following files:
		\begin{itemize} \itemsep 0pt
			\item \texttt{hackthefootline.sty} and
			\item \texttt{hackthefootline-doc.tex}
		\end{itemize}
	\end{small}
	
	
	\subsection*{Acknowledgement}
	\addcontentsline{toc}{subsection}{Acknowledgement}
	The basic layout of the two-- and three--column footline is based on the original \LaTeX\ beamer split and infolines outer themes written by Till Tantau.
	
	
	\subsection*{Notice: Behaviour inside the presentation's appendix}
	\addcontentsline{toc}{subsection}{Notice: Behaviour inside the presentation's appendix}
	By default, this package \emph{does not count appendix frames} into the total number of frames and resets the counter for frame numbers to 1 as soon as the \ltxcmd{\appendix} command is used. \\
	Also, frame numbers won't be displayed on frames inside the appendix.
	
	\medskip
	This behaviour can be avoided by passing the \texttt{countappendixframes} option to the package.
	
	
	
	\clearpage
	
	
	% DOCUMENTATION PART ------------------------------------------------------------------------------------------------------------------------
	
	\section{Using the package}
	The \mintinline{LaTeX}{\usepackage[<Options>]{hackthefootline}} command loads the package\footnote{To do so, the package has to be installed in a way \LaTeX\ is able to find it}. The complete list of options can be found in section \ref{packageoptions}.
	
	\medskip
	At least, you have to pass the \emph{footline's number of columns} as an option to the package \textit{(available: \texttt{onecol}, \texttt{twocols} or \texttt{threecols})}. If you don't do so, the footline gets removed.
	
	
	\section{Package options} \label{packageoptions}
	The \texttt{hackthefootline} package provides the following options:
	
	\medskip
	\DescribeMacro{onecol}
	The \enquote{initial footline hack} produces a one--column footline
	
	\medskip
	\DescribeMacro{twocols}
	The \enquote{initial footline hack} produces a two--column footline
	
	\medskip
	\DescribeMacro{threecols}
	The \enquote{initial footline hack} produces a three--column footline
	
	\medskip
	\DescribeMacro{countappendix-}\DescribeMacro{frames}
	If the \texttt{countappendixframes} option is passed to the package, appendix frames are counted into the total number of frames. \\
	\textit{Removes following dependencies: \texttt{appendixnumberbeamer}, \texttt{etoolbox}}
	
	\medskip
	\DescribeMacro{nofun}
	Doesn't support frame numbers being shown as \enquote{percent of presentation} \\
	\textit{Removes following dependencies: \texttt{calc}, \texttt{numprint}}
	
	
	\section{Default configurations}	
	\subsection*{One--column footline}
	\begin{tikzpicture}
		\draw (-.5\textwidth,1.25ex) rectangle (.5\textwidth,-1.25ex);
		\node[right] at (-.5\textwidth,0) {Short author, short institute:~~~~Short title};
		\node[left]  at ( .5\textwidth,0) {1 / 1};
	\end{tikzpicture}
	
	\subsection*{Two--column footline}
	\begin{tikzpicture}
		\draw (-.5\textwidth,1.25ex) rectangle (0,-1.25ex);
		\draw ( .5\textwidth,1.25ex) rectangle (0,-1.25ex);
		\node[left]  at (0,0) {Short author};
		\node[right] at (0,0) {Short title};
	\end{tikzpicture}
	
	\subsection*{Three-column footline}
	\begin{tikzpicture}
		\draw (-.5000\textwidth,1.25ex) rectangle (-.1667\textwidth,-1.25ex);
		\draw (-.1667\textwidth,1.25ex) rectangle ( .1667\textwidth,-1.25ex);
		\draw ( .1667\textwidth,1.25ex) rectangle ( .5000\textwidth,-1.25ex);
		\node[right] at (-.5\textwidth,0) {Short auth. (short inst.)};
		\node        at (0,            0) {Short title};
		\node[left]  at ( .5\textwidth,0) {Short date~~~~1/1};
	\end{tikzpicture}
	
	
	\section{Defined templates}
	The \texttt{hackthefootline} package defines templates with a variety of styles to provide easy personalisation. All defined templates are listed in table \ref{deftemplates}. Switching between different styles of a template is described in section \ref{switchingtemplates}.
	
	\begin{table} \centering\renewcommand{\arraystretch}{1.25}
		\begin{tabular}{lll}
			Template              & Style                               & Switching command             \\ \hline\hline
			\textbf{Title}        & \textit{Short title}                & \ltxcmd{\htfshorttitle}       \\
			                      & \textit{Long presentation title}    & \ltxcmd{\htflongtitle}        \\
			                      & \texttt{no title}                   & \ltxcmd{\htfnotitle}          \\ \hline
			\textbf{Author/Inst.} & \textit{Short author}               & \ltxcmd{\htfonlyauthor}       \\
			                      & \textit{Short institute}            & \ltxcmd{\htfonlyinstitute}    \\
			                      & \textit{Short author (short inst.)} & \ltxcmd{\htfinstitutepths}    \\
			                      & \textit{Short inst. (short author)} & \ltxcmd{\htfauthorpths}       \\
			                      & \textit{Short author, short inst.}  & \ltxcmd{\htfauthinst}         \\
			                      & \texttt{no author/inst.}            & \ltxcmd{\htfnoauthinst}       \\ \hline
			\textbf{Date}         & \textit{Short date}                 & \ltxcmd{\htfshortdate}        \\
			                      & \textit{Long date}                  & \ltxcmd{\htflongdate}         \\
			                      & \texttt{no date}                    & \ltxcmd{\htfnodate}           \\ \hline
			\textbf{Frame nr.}    & \textit{1}                          & \ltxcmd{\htfcounterframenrs}  \\
			                      & \textit{1 / 5}                      & \ltxcmd{\htffractionframenrs} \\
			                      & \textit{42\,\%}                     & \ltxcmd{\htfpercentframenrs}  \\
			                      & \texttt{no frame numbers}           & \ltxcmd{\htfnoframenrs}       \\ \hline
			\textbf{Separators:}  & \textit{Author: Title}              & \ltxcmd{\htfcolonsep}         \\
			                      & \textit{Author, Title}              & \ltxcmd{\htfcommasep}         \\
			                      & \textit{Author~~~~Title}            & \ltxcmd{\htfsepspace}
		\end{tabular}
		
		\caption{Defined templates and styles}
		\label{deftemplates}
	\end{table}
	
	\section{Switching styles} \label{switchingtemplates}
	The styles of the templates at the initial hack are selected depending on the footline's default settings. \\
	You are able to switch styles inside the \texttt{document} environment by \dots
	\begin{enumerate}
		\item changing the current settings using the commands listed in table~\ref{deftemplates} \textit{\textbf{and}}
		%
		\item applying the settings with the \ltxcmd{\hackthefootline} command. \par
			  \textbf{Your changes to the footline's settings only apply when executing this command.}
	\end{enumerate}
	
	\textit{Example:} \vspace{-.75em}
	\begin{minted}[gobble=2,tabsize=4]{LaTeX}
		\htflongdate		% Changes current settings ...
		\htfauthinst
		\hackthefootline	% ... and applies the changes
	\end{minted}
	
	\bigskip
	If a template is not shown in the current configuration (e.~g. the frame number in the two--column footline), just switch the style:

	\medskip
	\begin{tikzpicture}
		\draw (-.5\textwidth,1.25ex) rectangle (0,-1.25ex);
		\draw ( .5\textwidth,1.25ex) rectangle (0,-1.25ex);
		\node[left]  at (0,0) {Short author};
		\node[right] at (0,0) {Short title};
	\end{tikzpicture}
	\vspace{-1.5em}
	\begin{minted}[gobble=2,tabsize=4]{LaTeX}
		\htfcounterframenrs
		\hackthefootline
	\end{minted}
	\begin{tikzpicture}
		\draw (-.5\textwidth,1.25ex) rectangle (0,-1.25ex);
		\draw ( .5\textwidth,1.25ex) rectangle (0,-1.25ex);
		\node[left]  at (0,            0) {Short author};
		\node[right] at (0,            0) {Short title};
		\node[left]  at ( .5\textwidth,0) {1};
	\end{tikzpicture}
	
	
	\bigskip\bigskip
	Also, a template shown in the current configuration can be \enquote{removed} by switching the style:
	
	\medskip
	\begin{tikzpicture}
		\draw (-.5\textwidth,1.25ex) rectangle (0,-1.25ex);
		\draw ( .5\textwidth,1.25ex) rectangle (0,-1.25ex);
		\node[left]  at (0,            0) {Short author};
		\node[right] at (0,            0) {Short title};
		\node[left]  at ( .5\textwidth,0) {1};
	\end{tikzpicture}
	\vspace{-1.5em}
	\begin{minted}[gobble=2,tabsize=4]{LaTeX}
		\htfnoauthinst
		\hackthefootline
	\end{minted}
	\begin{tikzpicture}
		\draw (-.5\textwidth,1.25ex) rectangle (0,-1.25ex);
		\draw ( .5\textwidth,1.25ex) rectangle (0,-1.25ex);
		\node[right] at (0,            0) {Short title};
		\node[left]  at ( .5\textwidth,0) {1};
	\end{tikzpicture}
	
	\vfill
	\thispagestyle{empty}
	\listoftables
\end{document}