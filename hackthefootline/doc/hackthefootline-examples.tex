%%	This is file 'hackthefootline-examples.tex', Version 2018-01-16
%%	Copyright 2017 Sebastian Friedl <sfr682k@t-online.de>
%% 
%%	This work may be distributed and/or modified under the conditions of the LaTeX Project
%%	Public License, either version 1.3c of this license or (at your option) any later version.
%%	The latest version of this license is available at
%%		http://www.latex-project.org/lppl.txt
%%	and version 1.3c or later is part of all distributions of LaTeX version 2008-05-04 or later
%%
%%	This work has the LPPL maintenace status 'maintained'.
%%	The current maintainer of this work is Sebastian Friedl.
%%
%%	This work consists of the files hackthefootline.sty, hackthefootline-examples.tex and
%%	hackthefootline-doc.tex
%%
%%	-------------------------------------------------------------------------------------------
%%
%%	The hackthefootline package is a tool providing arbitrary footline selection and
%%	configuration for LaTeX beamer's standard themes (other themes may also work, of course)
%%
%%	-------------------------------------------------------------------------------------------
%%
%%	Please report bugs and other problems as well as suggestions for improvements
%%	to my email address (sfr682k@t-online.de).
%%
%%	-------------------------------------------------------------------------------------------
%%
%%	Acknowledgement:
%%	The basic layout of the two- and three-column footline is based on the original LaTeX
%%	beamer split and infolines outer themes written by Till Tantau.
%%
%%	-------------------------------------------------------------------------------------------


\documentclass[12pt]{beamer}

\usepackage[utf8]{inputenc}
\usepackage[tabular]{sourcesanspro}
\usepackage{sourcecodepro}

\usepackage[twocols]{hackthefootline}
\geometry{paperheight=1cm}

\usecolortheme{whale}
\usefonttheme[onlysmall]{structurebold}
\setbeamertemplate{headline}{}
\setbeamertemplate{navigation symbols}{}

\title{Examples for using the hackthefootline package}
\author{Sebastian Friedl <sfr682k@t-online.de>}

\begin{document}
	\title[Short title]{Title}
	\author[Short author]{Author}
	\institute[Short institute]{Institute}
	\date[Short date]{Date}
	
	\htfconfig{cols=one, title=short, authinst=both, date=none, framenrs=fraction, atsep=colon}
	\frame{}
	\addtocounter{framenumber}{-1}
	
	\htfconfig{cols=two, title=short, authinst=onlyauthor, date=none, framenrs=none}
	\frame{}
	\addtocounter{framenumber}{-1}
	
	\htfconfig{cols=three, title=short, authinst=instpths, date=short, framenrs=fraction}
	\frame{}
	\addtocounter{framenumber}{-1}
	
	
	\title[Short Introduction]{A short introduction on \texttt{hackthefootline}'s facilities}
	\author[S. Friedl]{Sebastian Friedl}
	\institute[Some Institute]{Some Institute far, far away}
	\date[2018/01/15]{January 15, 2018}
	\htfconfig{cols=two, title=short, authinst=onlyauthor, date=none, framenrs=none}
	\frame{}
	\addtocounter{framenumber}{-1}
	
	\htfconfig{framenrs=fraction}
	\frame{}
	\addtocounter{framenumber}{-1}
	
	\htfconfig{date=short}
	\frame{}
	\addtocounter{framenumber}{-1}
	
	\htfconfig{authinst=instpths}
	\frame{}
	\addtocounter{framenumber}{-1}
	
	\htfconfig{title=long}
	\frame{}
	\addtocounter{framenumber}{-1}
	
	\htfconfig{framenrs=none}
	\frame{}
	\addtocounter{framenumber}{-1}
	
	\htfconfig{cols=one, date=none}
	\frame{}
\end{document}
